\documentclass{beamer}
\usepackage[utf8]{inputenc}
\usepackage[spanish]{babel}
\usepackage{listings}
\lstloadlanguages{Ruby}
\lstset{%
  basicstyle=\ttfamily\color{black},
  commentstyle = \ttfamily\color{red},
  keywordstyle=\ttfamily\color{blue},
stringstyle=\color{orange}}

\begin{document}
\title{La primera abstracción funcional: lifting}
\author{Juan Pablo Santos}
\date{17 de octubre de 2015}

\begin{frame}
  \titlepage
\end{frame}

\begin{frame}
  \frametitle{Agenda}
  \tableofcontents
\end{frame}

\begin{frame}
  \frametitle{Funciones puras}
  \begin{itemize}
    \item Mismos argumentos, mismo resultado.
    \item Evaluación sin efectos secundarios implicitos.
    \item No mantiene estado implicito entre invocaciones.
    \item No es muy útil.
  \end{itemize}
\end{frame}

\begin{frame}
  \frametitle{El método Array\#map}
  \begin{itemize}
    \item ¿Qué hace?
    \item Nueva colección
    \item Efectos secundarios
    \item Transformación de colecciones
    \item Bloques
    \item vs \#each
  \end{itemize}

\end{frame}

\begin{frame}[fragile]
  \frametitle{El método \#map}
  \begin{itemize}
    \item Expectativas
    \begin{itemize}
        \item Identidad
        \item Composición
    \end{itemize}
  \end{itemize}

  \begin{lstlisting}[language=Ruby, caption=Identidad de \#map]
    [1,2,3].map { |n| n } == [1,2,3]
  \end{lstlisting}

  \begin{lstlisting}[language=Ruby, caption=Composición de \#map]
    [1,2,3].map { |n| n*n }.map { |m| m + 1 }
    ==
    [1,2,3].map { |n| (n*n) + 1 }
  \end{lstlisting}

\end{frame}

\begin{frame}[fragile]
  \frametitle{Adaptadores}
  \begin{itemize}
    \item Funciones sobre listas de valores.
    \item Funciones sobre valores.
    \item \#map como un adaptador. Funciones sobre valores a funciones
      sobre listas.
    \item ¿Listas vacias?
  \end{itemize}
\end{frame}

\begin{frame}[fragile]
  \frametitle{¿Otros adaptadores?}
  \begin{itemize}
    \item \#map once again!
    \item Funciones sobre valores que pueden ser nil.
    \item Funciones sobre valores.
    \item Possibly \ldots
  \end{itemize}
\end{frame}

\end{document}
